\documentclass[12pt, a4paper, oneside]{report}
\usepackage{graphicx}
\usepackage[utf8]{vietnam}
\usepackage{hyperref}
\usepackage[
    inner = 2.5cm,
    outer = 1.5cm,
    bottom=2cm,
    top=2.5cm
]{geometry}
\usepackage{setspace}
\setstretch{1.5}
\renewcommand{\baselinestretch}{1}
\usepackage{mlmodern}
\usepackage{fancyvrb}
\usepackage{fancyhdr, lastpage}
\usepackage{float}
\usepackage{tabularx}
\usepackage{etoolbox}
\usepackage{xcolor}
\usepackage{tikz}
\usepackage[most]{tcolorbox}
\usepackage[Conny]{fncychap}
\usepackage{listings}
\usepackage{titlesec}
\usepackage{amsmath}
\usepackage{array}
\usepackage{booktabs}
\usepackage{multirow}
\usepackage{tablefootnote}
\usepackage{graphicx}
\usepackage{subfigure}
\usepackage[table]{xcolor}
\usepackage{longtable}

\makeatletter
\patchcmd{\@makechapterhead}{\vspace*{50\p@}}{\vspace*{-5\p@}}{}{}
\makeatother
%Options: Sonny, Lenny, Glenn, Conny, Rejne, Bjarne, Bjornstrup
\setlength{\headheight}{32.34146pt} % Fix fancyhdr warning
\pagestyle{fancy}
\fancyhead{}
\lhead{\includegraphics[width=7cm]{bk_name_en.png}}
\cfoot{Page \thepage\ of \pageref{LastPage}}
\patchcmd{\chapter}{\thispagestyle{plain}}{\thispagestyle{fancy}}{}{}
% \usepackage{xcolor}
% \usepackage{mdframed}
\titleformat{\subsubsection}[runin] % runin keeps it on the same line
{\bfseries} % Format: bold text
{} % No prefix, you can include numbering here if needed
{0em} % Space between label and title text
{} % Code before title
\setcounter{secnumdepth}{3}

\begin{document}
{
\thispagestyle{empty} 
\centering
    \vspace*{1cm}
    
    \Large
    \textbf{ĐẠI HỌC QUỐC GIA THÀNH PHỐ HỒ CHÍ MINH TRƯỜNG ĐẠI HỌC BÁCH KHOA}
    
    \vspace{0.5cm}
    \Large
    \textbf{Khoa Khoa Học Và Kỹ Thuật Máy Tính}
    
    \vspace{0.5cm}
    
    \includegraphics[width=0.5\textwidth]{01_logobachkhoasang.png}\\
    \large
    \vspace{0.5cm}
    \textbf{Write-up Applied Cryptography}
    \vspace{0.5cm}
    \large
    \\
    \textbf{Name: Dương Bá Khang}
    \vspace{0.5cm}
    \\
    \textbf{Student ID: 2311403} 
    \vfill
    
    \normalsize
    Thành phố Hồ Chí Minh, tháng 9 năm 2025
    
}
\patchcmd{\chapter}{\thispagestyle{plain}}{\thispagestyle{fancy}}{}{}
\tableofcontents
\newpage

\chapter{Hash Function}
\section{Jack's Birthday Hash}
\textbf{Challenge:}

Today is Jack's birthday, so he has designed his own cryptographic hash as a way to celebrate.

Reading up on the key components of hash functions, he's a little worried about the security of the JACK11 hash.

Given any input data, JACK11 has been designed to produce a deterministic bit array of length 11, which is sensitive to small changes using the avalanche effect.

Using JACK11, his secret has the hash value: JACK(secret) = 01011001101.

Given no other data of the JACK11 hash algorithm, how many unique secrets would you expect to hash to have (on average) a 50\% chance of a collision with Jack's secret?

\textbf{Solve:}

Hash Value Output (11 bits contains 0 and 1) = $2^{11} = 2048$

So here A is the chance that we have a collision with Jack's secret.

$A = \frac{1}{2048}$ which the opposite is $\bar A = \frac{2047}{2048}$

So the chance of not colliding with Jack's secret is $\bar A$.

With 50\% chance of collision, we have:

$P(A) = 1 - P(\bar A) = 0.5$

$\Rightarrow P(\bar A) = 0.5$

$P(\bar A) = (\bar A)^n = (\frac{2047}{2048})^n = 0.5$

$\Rightarrow n * ln(\frac{2047}{2048}) = ln(0.5)$

$\Rightarrow n = \frac{ln(0.5)}{ln(\frac{2047}{2048})} \approx 1419.7$

So we need at least 1420 unique secrets to have a 50\% chance of a collision with Jack's secret.

\section{Jack's Birthday Confusion}
\textbf{Challenge:}

The last computation has made Jack a little worried about the safety of his hash, and after doing some more research it seems there's a bigger problem.

Given no other data of the JACK11 hash algorithm, how many unique secrets would you expect to hash to have (on average) a 75\% chance of a collision between two distinct secrets?

\textbf{Solve:}

$P(n) = 1 -$ (prob that n hashes are unique)

$\Rightarrow P(n) = 1 - (\frac{H}{H} * \frac{H-1}{H} * \frac{H-2}{H} * ... * \frac{H-n+1}{H})$
$\Rightarrow \Pi_{k = 0}^{n-1} (1 - \frac{k}{H})$

$\Rightarrow ln(\Pi_{k = 0}^{n-1} (1 - \frac{k}{H})) = \sum_{k = 0}^{n-1} ln(1 - \frac{k}{H})$ (1)

Where H is the number of hash value output = $2^{11} = 2048$

Talk about Taylor series, we have:
$ln (1-x) = -x - \frac{x^2}{2} - \frac{x^3}{3} - ...$

$\Rightarrow ln (1-x) \approx -x$ when x << 1

$\Rightarrow 1-x \approx e^{-x}$

So with the (1), we have:
$\sum_{k = 0}^{n-1} ln(1 - \frac{k}{H}) \approx - \sum_{k = 0}^{n-1} \frac{k}{H} = -\frac{n(n-1)}{2H}$ (2)

We take the exponential of both sides of (2) to get:
$\Pi_{k = 0}^{n-1} (1 - \frac{k}{H}) \approx e^{-\frac{n(n-1)}{2H}}$ (3)

But for the large n but still n << H, we can approximate n(n-1) $\approx$ n$^2$.

(3) becomes:
$e ^{-\frac{n^2}{2H}}$ 

$\Rightarrow P(n) \approx 1 - e^{-\frac{n^2}{2H}}$

$\Rightarrow - \frac{n^2}{2H} = ln(1 - p)$
$\Rightarrow n(p) \approx \sqrt{-2H * ln(1 - p)}$

$\Rightarrow n(0.75) \approx \sqrt{-2 * 2048 * ln(1 - 0.75)} \approx 76$

So there is 76 unique secrets to have a 75\% chance of a collision between two distinct secrets.
\end{document}