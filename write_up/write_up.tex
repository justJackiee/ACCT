\documentclass[12pt, a4paper, oneside]{report}
\usepackage{graphicx}
\usepackage[utf8]{vietnam}
\usepackage{hyperref}
\usepackage[
    inner = 2.5cm,
    outer = 1.5cm,
    bottom=2cm,
    top=2.5cm
]{geometry}
\usepackage{setspace}
\setstretch{1.5}
\renewcommand{\baselinestretch}{1}
\usepackage{mlmodern}
\usepackage{fancyvrb}
\usepackage{fancyhdr, lastpage}
\usepackage{float}
\usepackage{tabularx}
\usepackage{etoolbox}
\usepackage{xcolor}
\usepackage{tikz}
\usepackage[most]{tcolorbox}
\usepackage[Conny]{fncychap}
\usepackage{listings}
\usepackage{titlesec}
\usepackage{amsmath}
\usepackage{array}
\usepackage{booktabs}
\usepackage{multirow}
\usepackage{tablefootnote}
\usepackage{graphicx}
\usepackage{subfigure}
\usepackage[table]{xcolor}
\usepackage{longtable}

\makeatletter
\patchcmd{\@makechapterhead}{\vspace*{50\p@}}{\vspace*{-5\p@}}{}{}
\makeatother
%Options: Sonny, Lenny, Glenn, Conny, Rejne, Bjarne, Bjornstrup
\pagestyle{fancy}
\fancyhead{}
\lhead{\includegraphics[width=7cm]{bk_name_en.png}}
\cfoot{Page \thepage\ of \pageref{LastPage}}
\patchcmd{\chapter}{\thispagestyle{plain}}{\thispagestyle{fancy}}{}{}
% \usepackage{xcolor}
% \usepackage{mdframed}
\titleformat{\subsubsection}[runin] % runin keeps it on the same line
{\bfseries} % Format: bold text
{} % No prefix, you can include numbering here if needed
{0em} % Space between label and title text
{} % Code before title
\setcounter{secnumdepth}{3}

\begin{document}
{
\thispagestyle{empty} 
\centering
    \vspace*{1cm}
    
    \Large
    \textbf{ĐẠI HỌC QUỐC GIA THÀNH PHỐ HỒ CHÍ MINH TRƯỜNG ĐẠI HỌC BÁCH KHOA}
    
    \vspace{0.5cm}
    \Large
    \textbf{Khoa Khoa Học Và Kỹ Thuật Máy Tính}
    
    \vspace{0.5cm}
    
    \includegraphics[width=0.5\textwidth]{01_logobachkhoasang.png}\\
    \large
    \vspace{0.5cm}
    \textbf{Write-up Applied Cryptography}
    \vspace{0.5cm}
    \large
    \\
    \textbf{Name: Dương Bá Khang}
    \vspace{0.5cm}
    % \begin{table}[H]
    %     \begin{tabular}{|l|r|p{5cm}|}
    %         \vspace{0.5cm}
    %         \hline
    %         Cột 1 (căn trái) & Cột 2 (căn phải) & Cột 3 (căn đều) \\
    %         \hline
    %         Dữ liệu 1 & Dữ liệu 2 & Dữ liệu 3 được căn đều trong cột này và có thể tách dòng tự động nếu quá dài. \\
    %         Dữ liệu 4 & Dữ liệu 5 & Dữ liệu 6 cũng vậy. \\
    %         \hline
    %     \end{tabular}
    % \end{table}
    \\
    \textbf{Student ID: 2311403} 
    \vfill
    
    \normalsize
    Thành phố Hồ Chí Minh, tháng 9 năm 2025
    
}
\patchcmd{\chapter}{\thispagestyle{plain}}{\thispagestyle{fancy}}{}{}
\tableofcontents
\newpage

\chapter{Specification}
\begin{spacing}{1.5}
\section{Mục tiêu dự án}
\hspace{0.5cm}Dự án “Hệ thống hỗ trợ Tutor tại Trường ĐH Bách Khoa – ĐHQG TP.HCM (HCMUT)” nhằm:
\begin{enumerate}
    \item Quản lý và vận hành tập trung chương trình Tutor/Sinh viên của nhà trường một cách hiệu quả, hiện đại và có khả năng mở rộng.
    \item Cho phép quản lý hồ sơ Tutor/Sinh viên, đăng ký tham gia, lựa chọn hoặc gợi ý tutor, thiết lập và điều phối lịch hẹn (trực tiếp/online) kèm nhắc lịch tự động.
    \item Hỗ trợ theo dõi tiến độ – phản hồi – đánh giá để khoa/bộ môn, Phòng Đào tạo và Phòng CTSV khai thác dữ liệu, tối ưu phân bổ nguồn lực và xét điểm rèn luyện/học bổng.
\end{enumerate}
\hspace{0.5cm}Dự án có thể được mở rộng với việc tích hợp AI tùy theo nhu cầu và nguồn lực như (1) Ghép cặp tutor – sinh viên thông minh: hệ thống sử dụng kỹ thuật AI để gợi ý ghép cặp. (2) Cộng đồng trực tuyến cho tutor – mentee. (3) Chương trình tutor học thuật và phi học thuật.  (4) Hỗ trợ học tập cá nhân hóa

\section{Kết quả chính}
\hspace{0.5cm}Các kết quả chính của dự án bao gồm một nền tảng web đáp ứng đa thiết bị dành cho sinh viên, tutor và điều phối viên; mô-đun quản lý hồ sơ và quy trình đăng ký/duyệt tham gia chương trình; mô-đun đặt, hủy, dời lịch hẹn (trực tiếp hoặc trực tuyến) kèm hệ thống thông báo và nhắc lịch tự động; mô-đun thu thập phản hồi, thống kê tiến độ và sinh báo cáo phục vụ khoa, bộ môn, Phòng Đào tạo và Phòng CTSV; tích hợp hạ tầng sẵn có của HCMUT (SSO, DATACORE và thư viện số) để đồng bộ dữ liệu và cung cấp học liệu; cùng bộ tài liệu đặc tả, thiết kế, giao diện mẫu, kịch bản kiểm thử và phiên bản MVP hoàn chỉnh cho luồng đăng nhập – đăng ký – đặt lịch – phản hồi.

\section{Tiêu chuẩn kỹ thuật}
\begin{enumerate}
    \item Xác thực và Phân quyền: bắt buộc tích hợp HCMUT-SSO; phân quyền dựa trên vai trò (student / tutor / coordinator / department chair / admin) đồng bộ từ HCMUT-DATACORE.
    \item Đồng bộ dữ liệu: sử dụng dịch vụ chia sẻ dữ liệu của DATACORE để cập nhật hồ sơ người dùng theo thời gian thực, giảm nhập liệu thủ công.
    \item Tài liệu học liệu: kết nối HCMUT-LIBRARY để tra cứu – chia sẻ giáo trình số chính thống cho mỗi buổi tư vấn.
    \item Kiến trúc và Triển khai: Ứng dụng web-based (SPA hoặc MPA), REST/GraphQL API, database quan hệ (PostgreSQL/MySQL) và container hóa (Docker). Công nghệ cụ thể sẽ được lựa chọn dựa trên tiêu chí bảo mật, khả năng mở rộng và mức độ hỗ trợ của hạ tầng trường.
    \item Bảo mật và Tuân thủ: truyền dữ liệu qua HTTPS, tuân thủ quy định bảo vệ dữ liệu cá nhân; logging toàn bộ phiên đăng nhập, thao tác nhạy cảm.
\end{enumerate}

\end{spacing}

\end{document}